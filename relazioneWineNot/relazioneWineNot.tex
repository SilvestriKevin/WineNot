\input{StyleLatex/global.tex}
\input{StyleLatex/layout.tex}
\input{res/local}

\title{\textbf{Relazione progetto TecWeb}}
\author{WineNot}

\date{1 Febbraio 2018}

\begin{document}

%\maketitle

\makeFrontPage

\tableofcontents

\newpage

\section{Abstract}

Lo scopo del progetto sviluppato è quello di implementare un sito web per un'enoteca, nella quale un utente può visualizzare tutti i vini disponibili, le migliori annate, gli eventi e la storia dell'enoteca. 
Il sito ha la funzione di vetrina per l'enoteca WineNot e serve per pubblicizzarla. \\
L'utente che visita il sito può accedere alle pagine \textit{Home}, \textit{Vini}, \textit{Vino specifico}, \textit{Annate migliori}, \textit{Eventi}, \textit{Storia} e \textit{Contattaci}:
\begin{itemize}
	\item \textbf{Home}: in questa pagina l'utente può trovare alcune informazioni riguardanti l'enoteca, la sua storia, gli eventi e le tipologie di vini da essa offerta. Cliccando sul titolo di una tipologia di vini, si viene trasportati nella pagina di ricerca, in cui verranno elencati i vini desiderati; 
	\item \textbf{Vini}: in questa pagina è predisposto un form di ricerca per trovare velocemente il vino desiderato. Il form contiene, oltre alla barra di 
	input testuale, un filtro dei risultati per \textit{annata} e \textit{tipologia}, oltre alla possibilità di ordinare i risultati di ricerca per \textit{nome}, \textit{annata}, \textit{tipologia} e \textit{gradazione}. Per ciascun vino vengono visualizzati il \textit{nome}, la \textit{tipologia}, l'\textit{annata} e la \textit{gradazione}. Cliccando l'immagine del vino sarà possibile visualizzare le informazioni dettagliate per quest'ultimo;
	\item \textbf{Annate Migliori}: in questa pagina l'utente potrà visualizzare informazioni sulle annate contrassegnate come migliori, ed i vini ad esse associati. La visualizzazione di questi ultimi rispecchia quella vista nella pagina di ricerca Vini, mostrando quindi \textit{nome}, \textit{tipologia} e \textit{gradazione} per ognuno di essi;
	\item \textbf{Vino specifico}: in questa pagina, oltre all'immagine relativa al vino in questione, si potranno visualizzare tutte le informazioni necessarie, suddivise in \textit{Dettagli} (nome, denominazione, tipologia, vitigno e annata), \textit{Piatti e Occasioni} (abbinamento e degustazione) e \textit{Quantità} (formato bottiglia e gradazione). Vi è inoltre presente un link che porterà alla pagina precedentemente visualizzata, in modo da non far perdere l'utente visitatore all'interno del sito;
	\item \textbf{Eventi}: in questa pagina si potranno visualizzare gli eventi offerti al pubblico periodicamente. Inoltre, è presente un link alla pagina \textit{Contattaci}, in modo che gli utenti visitatori possano contattare WineNot in caso di domande particolari riguardanti gli eventi;
	\item \textbf{Storia}: in questa pagina vi sono presenti diverse sezioni che contengono informazioni sulla storia dell'enoteca WineNot e sulla famiglia che la gestisce;
	\item \textbf{Contattaci:} in questa pagina sarà possibile per l'utente visitatore contattare WineNot utilizzando l'apposito form nella parte alta della pagina, i contatti telefonici, email o fax, oppure presentandosi di persona utilizzando la posizione geografica.
\end{itemize}

Ci sono poi l'amministratore (unico) e i collaboratori. Entrambi i ruoli possono aggiungere/modificare/eliminare vini e annate e anche modificare i propri dati profilo. Solo l'amministratore potrà aggiungere/modificare/eliminare i collaboratori.
\textbf{Per accedere alla pagina di \textit{login} è sufficiente scrivere \textit{/admin} in coda all'url della pagina e premere invio.} In quest'ultima pagina si può anche recuperare la propria password, utilizzando la mail associata all'account.\\
Accedendo è possibile visualizzare quattro sezioni: \textit{gestione vini}, \textit{gestioni annate}, \textit{gestione utenti} e \textit{dati profilo}:

\begin{itemize}
	\item \textbf{Gestione Vini:} in questa pagina viene riproposto il form di ricerca presente nella pagina \textit{Vini}. Questi ultimi vengono proposti in forma tabellare e vengono accompagnati da alcuni bottoni che permettono di selezionare/deselezionare i vini ed eliminare quelli attualmente selezionati. Per ciascun vino elencato in tabella, vi sono presenti gli attributi \textit{nome}, \textit{denominazione}, \textit{tipologia} e \textit{annata}. Inoltre vi è possibile accedere alle seguenti pagine:
		\begin{itemize}
			\item \textbf{Aggiungi Vino:} è presente un form attraverso il quale bisogna riempire tutti i campi dati associati ad un vino, compresa l'immagine in formato \textit{'.png'}. In caso non esista già l'annata desiderata, è presente un link diretto alla pagina \textit{Aggiungi Annata}, che verrà spiegata più avanti. \textbf{Tutti} i cambi dati sono obbligatori;
			\item \textbf{Modifica Vino:} è presente un form, contenente al suo interno tutte le informazioni relative al vino richiesto. Sarà quindi possibile modificare \textbf{tutti} quanti i cambi dati inclusa l'annata la quale, se non presente, potrà essere aggiunta utilizzando il link diretto alla pagina \textit{Aggiungi Annata};
			\item \textbf{Elimina Vino:} vi è presente il vino (o vini, in caso di selezione multipla) che si vuole eliminare. Si potrà quindi confermare l'eliminazione oppure annullarla, tornando indietro alla pagina precedente.
		\end{itemize}
	
	\item \textbf{Gestione Annate:} in questa pagina vengono elencate, in forma tabellare, le annate presenti all'interno del database. Anche qui vengono proposti i bottoni per selezionare/deselezionare tutte le annate ed eliminare quelle selezionate. Gli attributi presenti in tabella sono \textit{annata}, \textit{qualità} e \textit{migliore}. E' possibile inoltre accedere alle pagine:
		\begin{itemize}
			\item \textbf{Aggiungi Annata:} è presente il form attraverso il quale si potrà aggiungere una nuova annata. Anche qui \textbf{tutti} gli attributi sono obbligatori;
			\item \textbf{Modifica Annata:} lo stesso form di \textit{Aggiungi Annata} viene anche qui proposto, questa volta contenente le informazioni dell'annata in questione. Tuttavia, se si vuole cambiare il campo dato \textit{anno}, non sarà possibile farlo quando ci sono già vini associati ad esso. Questo per assicurare che non vengano cancellati/modificati, per errore, anche i vini associati all'annata stessa. Il \textit{collaboratore}/\textit{amministratore} viene quindi obbligato a dover prima cambiare/eliminare i vini associati e successivamente cambiare l'annata, o in casi particolari, aggiungerne una nuova;
			\item \textbf{Elimina Annata:} vi è presente l'annata (o le annate, in caso di selezione multipla) che si vuole eliminare. Si potrà quindi confermare l'eliminazione oppure annullarla, tornando indietro alla pagina precedente.
		\end{itemize}
	\item \textbf{Gestione Utenti:} \textbf{in caso si fosse amministratore}, si potrà visualizzare la lista di collaboratori in formato tabellare, come già visto in precedenza. Gli attributi visualizzabili per ciascun utente sono \textit{nome completo}, \textit{username} ed \textit{email}. Anche qui vengono proposti i bottoni per selezionare/deselezionare tutti gli utenti ed eliminare quelli selezionati. Vi è inoltre la possibilità di accedere alle pagine:
		\begin{itemize}
			\item \textbf{Aggiungi Utente:} attraverso un form sarà possibile per l'amministratore aggiungere un nuovo utente, compilando \textbf{tutti} i campi richiesti;
			\item \textbf{Modifica Utente:} lo stesso form viene ripresentato anche in questa pagina, per poter modificare i dati di uno specifico utente. Tuttavia, se non si vuole cambiare la password del collaboratore, i campi \textit{password nuova} e \textit{conferma password} non sono obbligatori;
			\item \textbf{Elimina Utente:} vi è presente l'utente (o gli utenti, in caso di selezione multipla) che si vuole eliminare. Si potrà quindi confermare l'eliminazione oppure annullarla, tornando indietro alla pagina precedente.
		\end{itemize}
	\item \textbf{Dati Profilo:} in questa pagina è presente un form per la modifica dei dati del proprio profilo, accessibile sia ai \textit{collaboratori} che all'\textit{amministratore}. Se si vogliono modificare solo il \textit{nome completo}, l'\textit{username} o l'\textit{email}, non sono richiesti i campi dati \textit{password attuale} e \textit{password nuova}.
\end{itemize}

\subsection{Utenti destinatari}

Il sito è destinato a qualsiasi utente il cui scopo sia quello di consultare
i vini offerti dall'enoteca WineNot, gli eventi organizzati ed eventualmente contattare l'enoteca per informazioni aggiuntive. 

\subsection{Gestione dei dati}

Il sito presenta dei contenuti modificabili da parte dei collaboratori e dell'amministratore. 
Qui di seguito sono riportate nello specifico le modifiche possibili da parte di ciascuna categoria di utilizzatore:

\subsubsection{Collaboratore}

Previo login può visualizzare, aggiungere, modificare o rimuovere i vini e le annate. 
Una volta effettuato l'accesso all'area personale, può modificare le sue
informazioni personali, quali nome completo, username, email e password.

\subsubsection{Amministratore}

L'amministratore può fare tutto ciò che può fare un collaboratore, oltre a 
visualizzare, aggiungere, modificare o rimuovere i collaboratori.


\section{Fasi di progettazione}

Lo sviluppo del progetto si è suddiviso in varie fasi, raggruppabili nelle seguenti categorie:

\subsection{Analisi dei requisiti}

Il gruppo si è più volte riunito per analizzare ( e progettare ) dei requisiti che potessero soddisfare ed essere il più vicini possibili alla realtà di un' enoteca. 
I requisiti presi in considerazione sono stati bacino di utenza, funzionalità
e la visualizzazione ed interazione delle pagine web da parte degli utenti. 
Il gruppo ha voluto inoltre dare una separazione tra collaboratori e amministratore, con permessi di modifica maggiori nel sito web. 

\subsection{Realizzazione ed implementazione delle funzionalità}

Ogni funzionalità implementata nel sito è frutto di un attento studio attraverso esclusive sessioni di lavoro.
Il sito web si appoggia innanzitutto su una base di dati dove si vanno a memorizzare tutte le informazioni viste a video, quali dati sui vini, annate e utenti. 
Le informazioni su questa base di dati vengono ricercate, estrapolate e rielaborate attraverso speciali pagine, al cui interno contengono delle
query che interrogano direttamente il database attraverso una connessione dedicata.
Particolare attenzione è stata inoltre riservata alla visualizzazione del sito su dispositivi mobile, dove risulta ottimizzata ogni pagina direttamente interagibile dall'utente.

\subsection{Fase di test}

Ogni funzionalità implementata nel sito web è stata oggetto di numerosi test per verificarne il suo effettivo funzionamento. Gli strumenti utilizzati sono stati \textbf{W3C Markup Validation Service} (per validazione XHTML), \textbf{W3C CSS Validator} (per validazione CSS), \textbf{browserstack.com} e plugin per Chrome \textbf{IE-tab} (per test su diversi browser).

\section{Struttura del progetto}

Causa enorme complessità del progetto, i file che compongono il sito sono stati
raggruppati e situati in 7 diverse cartelle:

\begin{itemize}
	\item \textbf{css:} questa cartella contiene tutti i file riguardanti il css, opportunamente diviso in tre file, per desktop, mobile e stampa;
	\item \textbf{html:} questa cartella contiene i file html del sito web;
	\item \textbf{img:} questa cartella contiene tutte le immagini dei vini del sito web;
	\item \textbf{include:} questa cartella contiene il file di connessione al database e il file con le funzioni ausiliarie utilizzate nelle pagine php del sito web;
	\item \textbf{js:} questa pagina contiene tutti gli script javascript del progetto, quali "javascript.js" (controlli sull'input) e 
	"mappa.js" (script per visualizzare la mappa sulla pagina "Contattaci" del sito web) \textbf{[RICORDARE DI AGGIUNGERE FONTI];}
	\item \textbf{php:} questa cartella contiene i file php del sito web;
	\item \textbf{utility:} questa cartella contiene un eseguibile utile per ricevere le mail che sarebbero spedite dal sito web, in modo da poter verificare il corretto funzionamento del recupero password.
\end{itemize}

\section{Gestione dei dati}

Il sito web si appoggia su una base di dati di tipo MySQL. Il database immagazzina tutti i dati inseriti negli appositi form del sito web, rendendoli disponibili ad ogni query eseguita in una pagina. Tutte le modifiche che avvengono tramite query MySQL vengono lanciate da pagine web PHP. La struttura del database è come segue:

\begin{itemize}
	\item \textbf{Annata:} in questa tabella vengono immagazzinate tutte le
	informazioni sulle annate dei vini presenti nell'enoteca. La tabella è strutturata come segue:
	\begin{itemize}
		\item \textbf{Anno:} indica l'anno ed è univoco;
		\item \textbf{Descrizione:} descrive l'annata;
		\item \textbf{Qualità:} descrive la qualità dell'annata;
		\item \textbf{Migliore:} indica se fa parte delle migliori annate dell'enoteca.
	\end{itemize}	
	\item \textbf{Utenti:} in questa tabella vengono immagazzinate tutte le informazioni sugli utenti presenti nell'enoteca. La tabella è strutturata come segue:
	\begin{itemize}
		\item \textbf{Id\_user:} indica l'id dell'utente ed è univoco;
		\item \textbf{Nome:} indica il nome completo;
		\item \textbf{Username:} indica l'username per la login;
		\item \textbf{Password:} indica la password per la login;
		\item \textbf{Email:} indica l'email;
		\item \textbf{Admin:} indica se l'utente è l'amministratore.
	\end{itemize}
	\item \textbf{Vini:} in questa tabella vengono immagazzinate tutte le informazioni sui vini presenti nell'enoteca. La tabella è strutturata come segue:
	\begin{itemize}
		\item \textbf{Id\_wine:} indica l'id del vino ed è univoco;
		\item \textbf{Nome:} indica il nome;
		\item \textbf{Tipologia:} indica la tipologia;
		\item \textbf{Descrizione:} indica la descrizione;
		\item \textbf{Denominazione:} indica la denominazione;
		\item \textbf{Annata:} indica l'annata;
		\item \textbf{Vitigno:} indica i vitigni con cui è realizzato il vino;
		\item \textbf{Abbinamento:} indica gli abbinamenti;
		\item \textbf{Degustazione:} indica quando è consigliato degustare il vino;
		\item \textbf{Gradazione:} indica la gradazione alcolica (in percentuale);
		\item \textbf{Formato:} indica il formato (in litri) della bottiglia.
	\end{itemize}
\end{itemize}

\section{Struttura sito}

Alcune funzionalità del sito sono usufruibili previo login. Qui di
seguito sono elencati tutti i livelli di accessibilità possibili per il sito, raggruppati secondo le seguenti nomenclature:
\begin{itemize}
	\item \textbf{Everyone}: identifica qualsiasi utente;
	\item \textbf{Administrator}: identifica l'amministratore che ha eseguito il login al sito web;
	\item \textbf{Contributor}: identifica qualsiasi collaboratore che ha eseguito il login al sito web.
\end{itemize}




\subsection{Funzionalità}

\begin{table}[H]
	\centering
	\begin{tabular}{c|c|C{4cm}}
		\rowcolorhead
		\rowheadercell{\textbf{Funzionalità}} & \rowheadercell{\textbf{Livello di accesso}} \\
		\rowcolorlight
		\textit{Esplorazione del sito}  & Everyone \\
		\rowcolordark
		\textit{Ricerca vini}  & Everyone \\
		\rowcolorlight
		\textit{Invio richiesta}  & Everyone \\
		\rowcolordark
		\textit{Aggiunta, Modifica, Eliminazione vino}  & Contributor, Administrator \\
		\rowcolorlight
		\textit{Aggiunta, Modifica, Eliminazione annata}  & Contributor, Administrator \\
		\rowcolordark
		\textit{Aggiunta, Modifica, Eliminazione utente}  & Administrator \\
		\rowcolorlight
		\textit{Modifica Dati Profilo}  & Contributor, Administrator \\
		
	\end{tabular}
	\caption{Funzionalità e relativi Livelli di accesso per il sito \textit{WineNot}}
\end{table}


\section{Adattabilità ed usabilità del sito}

La correttezza della lettura del sito web è facilitata dall'introduzione di elementi
tabindex, e la forte presenza di colori con contrasto elevato facilita la lettura del
sito web anche da persone con capacità visive limitate. Il sito web inoltre è realizzato 
tramite layout responsive e nel completo rispetto dei vigenti standard di
accessibilità.\\
Non risulta inoltre mai necessario alcuno scorrimento orizzontale in quanto il sito
web si adatta alla dimensione di ogni schermo da dove viene visualizzato. Nel
sito web sono presenti alcuni elementi per aiutare l'utente nella navigazione, come ad esempio i pulsanti "torna su" e "torna indietro" qualora fosse necessario. Non sono inoltre
presenti contenuti a rischio di epilessia o che possano indurre l'utente in uno stato
confusionale, grazie alla scelta da parte del nostro team di utilizzare un design semplice. \\
Abbiamo inoltre limitato l'utilizzo di Javascript ai soli messaggi di errore nei diversi form presenti sul sito. Questa scelta è stata fatta con lo scopo di rendere il passaggio dal sito con Javascript disattivato a Javascript attivato il meno sconvolgente possibile per l'utente finale.

\section{Compatibilità browser}

Il presente sito web è stato testato su vari browser in varie piattaforme tramite l'utilizzo del sito \textbf{browserstack.com} e il plugin per Chrome \textbf{IE-tab}. Dai test effettuati, non sono emerse problematiche che possano influire negativamente sulla navigazione o sull'usabilità del sito.
Di seguito si riporta un breve riepilogo delle principali piattaforme testate:
	\begin{itemize}
		\item \textbf{Internet Explorer 9.0}: il sito viene visualizzato ed è interagibile correttamente;
		\item \textbf{Firefox 35}: il sito viene visualizzato ed è interagibile correttamente;
		\item \textbf{Chrome 64}: il sito viene visualizzato ed è interagibile correttamente;
		\item \textbf{Chrome 38}: il sito viene visualizzato ed è interagibile correttamente anche se, tuttavia, il form non contengono il bordo, come succede negli altri browser;
		\item \textbf{Safari iOS}: il sito viene visualizzato ed è interagibile correttamente;
		\item \textbf{Opera 19}: il sito viene visualizzato ed è interagibile correttamente.
	\end{itemize}

\section{Organizzazione del gruppo di lavoro}

La seguente tabella indica quali sono state le principali mansioni di ogni componente del gruppo. Ogni componente, tuttavia, ha contribuito almeno in parte ad ogni fase di sviluppo del progetto.

\subsection{Pirlog Cristian}

\begin{itemize}
	\item HTML
	\item Javascript
	\item PHP: controlli dei form e pagine del pannello di amministrazione
	\item Gestione sessioni
	\item Gestione connessioni
	\item Validazione
	\item Accessibilità
	\item Stesura relazione
\end{itemize}

\subsection{Silvestri Kevin}
\begin{itemize}
	\item Creazione database
	\item HTML
	\item PHP: creazione delle query, controlli dei form, pagine del pannello di amministrazione
	\item Ricerca vini
	\item Gestione login
	\item Gestione recupero password
	\item Gestione errori
	\item Validazione
	\item Stesura relazione
\end{itemize}

\subsection{Thiella Eleonora}

\begin{itemize}
	\item CSS Desktop
	\item CSS Mobile
	\item CSS Stampa
	\item HTML
	\item Accessibilità
\end{itemize}

\end{document}