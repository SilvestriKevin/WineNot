\input{StyleLatex/global.tex}
\input{StyleLatex/layout.tex}
\input{res/local}

\title{\textbf{Relazione progetto TecWeb}}
\author{WineNot}

\date{1 Febbraio 2018}

\begin{document}

%\maketitle

\makeFrontPage

\tableofcontents

\newpage

\section{Abstract}

Lo scopo del progetto sviluppato è quello di implementare un sito web per un'enoteca,
nella quale un utente può visualizzare tutti i vini disponibili, le migliori annate, 
gli eventi e la storia dell'enoteca. 
Il sito ha la funzione di vetrina per l'enoteca WineNot e serve per pubblicizzarla. 
L'utente che visita il sito può accedere a Vini, Annate migliori, Eventi e Storia. 
Nella pagina dei vini è predisposto un form di ricerca per trovare velocemente il 
vino desiderato e cliccando l'immagine del vino visualizzare le informazioni dettagliate.
Nella pagina delle migliori annate è possibile visualizzare i vini ralativi alle annate.
Nella pagina eventi sono descritti gli eventi annuali.
Nella pagina storia è descritta la storia dell'enoteca.
L'utente può poi contattare l'enoteca per avere informazioni tramite l'apposito modulo nella pagina Contattaci.
Ci sono poi l'amministratore (unico) e i collaboratori. Entrambi i ruoli possono aggiungere/modificare/eliminare 
vini e annate e anche modificare i propri dati profilo. Solo l'amministratore può 
aggiungere/modificare/eliminare i collaboratori.
Per accedere alla pagina di login è sufficiente scrivere /admin in coda all'url della pagina e premere invio.
Accedendo è possibile visualizzare quattro sezioni: gestione vini, gestioni annate, gestione utenti e dati profilo.


\subsection{Utenti destinatari}

Il sito è destinato a qualsiasi utente il cui scopo è quello di consultare
i vini offerti dall'enoteca WineNot, gli eventi organizzati ed eventualmente contattare 
l'enoteca per informazioni aggiuntive. 

\subsection{Gestione dei dati}

Il sito presenta dei contenuti modificabili da parte dei collaboratori e dell'amministratore. 
Qui di seguito sono riportati nello specifico le modifiche possibili da parte di ciascuna categoria di utilizzatore:

\subsubsection{Collaboratore}

Previo login può visualizzare, aggiungere, modificare o rimuovere i vini e le annate. 
Una volta effettuato l'accesso all'area personale, può modificare le sue
informazioni personali, quali nome completo, username, email e password.

\subsubsection{Amministratore}

L'amministratore può fare tutto ciò che può fare un collaboratore, oltre a 
visualizzare, aggiungere, modificare o rimuovere i collaboratori.


\section{Fasi di progettazione}

Lo sviluppo del progetto si è suddiviso in varie fasi, raggruppabili nelle seguenti
categorie:

\subsection{Analisi dei requisiti}

Il gruppo si è più volte riunito per analizzare ( e progettare ) dei requisiti che
potessero soddisfare ed essere il più vicini possibili alla realtà di una enoteca. 
I requisiti presi in considerazione sono stati bacino di utenza, funzionalità,
e la visualizzazione ed interazione delle pagine web da parte degli utenti. 
Il gruppo ha voluto inoltre dare una separazione tra collaboratori e amministratore, 
che permessi di modifica maggiori nel sito web. 

\subsection{Realizzazione ed implementazione delle funzionalità}

Ogni funzionalità implementata nel sito è frutto di un attento studio attraverso
esclusive sessioni di lavoro.
Il sito web si appoggia innanzitutto su una base di dati dove si vanno a memorizzare 
tutte le informazioni viste a video, quali dati sui vini, annate e utenti. 
Le informazioni su questa base di dati vengono ricercate,
estrapolate e rielaborate attraverso speciali pagine, al cui interno contengono delle
query che interrogano direttamente il database attraverso una connessione dedicata.
Particolare attenzione è stata inoltre riservata alla visualizzazione del sito su
dispositivi mobile, dove risulta ottimizzata ogni pagina direttamente interagibile
dall'utente.

\subsection{Fase di test}

Ogni funzionalità implementata nel sito web è stata oggetto di numerosi test per
vericarne il suo effettivo funzionamento. Gli strumenti utilizzati sono stati W3C
Markup Validator, Total Validator, XHTML validator e browserstack.com.

\section{Struttura del progetto}

Causa enorme complessità del progetto, i file che compongono il sito sono stati
raggruppati e situati in 7 diverse cartelle:

\begin{itemize}
	\item \textbf{CARTELLA "css":} Questa cartella contiene tutti i file riguardanti il css, 
	opportunamente diviso in tre file, per desktop, mobile e stampa;
	\item \textbf{CARTELLA "html":} Questa cartella contiene i file html del sito web;
	\item \textbf{CARTELLA "img":} Questa cartella contiene tutte le immagini dei vini del sito web;
	\item \textbf{CARTELLA "include":} Questa cartella contiene il file di connessione al database e 
	il file con le funzioni ausiliarie utilizzate nelle pagine php del sito web;
	\item \textbf{CARTELLA "js":} Questa pagina contiene tutti gli script javascript
	del progetto, quali "javascript.js" (script per slideshow e controlli sull'input) e 
	"mappa.js" (script per visualizzare la mappa sulla pagina "Contattaci" del sito web);
	\item \textbf{CARTELLA "php":} Questa cartella contiene i file php del sito web;
	\item \textbf{CARTELLA "utility":} Questa cartella contiene un eseguibile utile per ricevere 
	le mail che sarebbero spedite dal sito web e questo per verificare il corretto funzionamento 
	del recupero password;
\end{itemize}

\section{Gestione dei dati}

Il sito web si appoggia su una base di dati di tipo MySQL. Il database immagazzina
tutti i dati inseriti negli appositi form del sito web, rendendoli disponibili ad ogni
query eseguita in una pagina. Tutte le modiche avvengono tramite query MySQL
lanciate da pagine web PHP. La struttura del database è come segue:

\begin{itemize}
	\item \textbf{TABELLA "Annata":} in questa tabella vengono immagazzinate tutte le
	informazioni sulle annate dei vini presenti nell'enoteca. La tabella è strutturata come segue:
	\begin{itemize}
		\item \textbf{Anno:} indica l'anno ed è univoco;
		\item \textbf{Descrizione:} descrive l'annata;
		\item \textbf{Qualita:} descrive la qualità;
		\item \textbf{Migliore:} indica se fa parte delle migliori annate dell'enoteca.
	\end{itemize}	
	\item \textbf{TABELLA "Utenti":} in questa tabella vengono immagazzinate tutte le
	informazioni sugli utenti presenti nell'enoteca. La tabella è strutturata
	come segue:
	\begin{itemize}
		\item \textbf{Id\_user:} indica l'id dell'utente ed è univoco;
		\item \textbf{Nome:} indica il nome completo;
		\item \textbf{Username:} indica l'username per la login;
		\item \textbf{Password:} indica la password per la login;
		\item \textbf{Email:} indica l'email;
		\item \textbf{Admin:} indica se l'utente è l'amministratore.
	\end{itemize}
	\item \textbf{TABELLA "Vini":} in questa tabella vengono immagazzinate tutte le
	informazioni sui vini presenti nell'enoteca. La tabella è strutturata
	come segue:
	\begin{itemize}
		\item \textbf{Id\_wine:} indica l'id del vino ed è univoco;
		\item \textbf{Nome:} indica il nome;
		\item \textbf{Tipologia:} indica la tipologia;
		\item \textbf{Descrizione:} indica la descrizione;
		\item \textbf{Denominazione:} indica la denominazione;
		\item \textbf{Annata:} indica l'annata;
		\item \textbf{Vitigno:} indica i vitigni con cui è realizzato il vino;
		\item \textbf{Abbinamento:} indica gli abbinamenti;
		\item \textbf{Degustazione:} indica quando è consigliato degustare il vino;
		\item \textbf{Gradazione:} indica la gradazione;
		\item \textbf{Formato:} indica il formato della bottiglia.
	\end{itemize}
\end{itemize}

\section{Struttura sito}

Alcune funzionalità del sito sono usufruibili previo login. Qui di
seguito sono elencate tutte le funzionalità del sito, raggruppate secondo le seguenti
nomenclature:
Everyone: identifica qualsiasi utente.
Administrator: identifica l'amministratore che ha eseguito il login al sito web.
Contributor: identica qualsiasi collaboratore che ha eseguito il login al sito web.

\subsection{Funzionalità utenza usufruibile}

\begin{center}
	\begin{tabular}{r|l}
		Esplorazione del sito & Everyone \\
		Ricerca vini & Everyone \\
		Invio richiesta & Everyone \\
		Aggiunta, Modifica , Eliminazione vino & Contributor, Administrator \\
		Aggiunta, Modifica , Eliminazione annata & Contributor, Administrator \\
		Aggiunta, Modifica , Eliminazione utente & Administrator \\
		Modifica dati profilo & Contributor, Administrator \\
		... & ... \\
	\end{tabular}
\end{center}

\section{Adattabilità ed usabilità del sito}

La correttezza della lettura del sito web è facilitata dall'introduzione di elementi
tabindex, e la forte presenza di colori con contrasto elevato facilita la lettura del
sito web anche da persone con capacità visive limitate. Il sito web inoltre è realizzato 
tramite layout responsive e nel completo rispetto dei vigenti standard di
accessibilità.
Non risulta inoltre mai necessario alcuno scorrimento orizzontale in quanto il sito
web si adatta alla dimensione di ogni schermo da dove viene visualizzato. Nel
sito web sono presenti alcuni elementi per aiutare l'utente nella navigazione, quali
colori diversi per ogni link visitato, mappa del sito e pulsante "torna su". Non sono inoltre
presenti contenuti a rischio di epilessia o che possano indurre l'utente in uno stato
confusionale.

\section{Compatibilità browser}

Il presente sito web è stato testato su vari browser in varie piattaforme tramite l'utilizzo del sito browserstack.com. Dai test effettuati, non sono emerse problematiche che possano in
uire negativamente sulla navigazione o sullusabilità del sito.
Di seguito si riporta un breve riepilogo delle principali piattaforme testate:
-Internet Explorer 8.0: Il sito è visualizzato ed interagibile correttamente -
Firefox 30: Il sito è visualizzabile ed interagibile correttamente -Safari iOS: Il sito
è visualizzabile ed interagibile correttamente -Google Chrome 60: Il sito è visual-
izzabile ed interagibile correttamente -Google Chrome 40: Il sito è visualizzabile
ed interagibile correttamente -Opera 12.12: Il sito è visualizzabile ed interagibile
correttamente

\section{Organizzazione del gruppo di lavoro}

La seguente tabella indica quali sono state le principali mansioni di ogni componente
 del gruppo. Ogni componente, tuttavia, ha contribuito almeno in parte ad
ogni fase di sviluppo del progetto.

\subsection{Pirlog Cristian}

\begin{itemize}
	\item HTML
	\item Javascript
	\item Aiuto query php, parte controlli php ed aiuto pagine php utenti
	\item PHP: gestione sessioni e gestione connessioni
	\item Validazione
	\item Accessibilità
\end{itemize}

\subsection{Silvestri Kevin}

\begin{itemize}
	\item HTML
	\item Validazione
	\item Creazione database
	\item PHP
	\item Ricerca vini
	\item Controlli PHP
	\item Stesura relazione
	\item Gestione login
	\item Gestione recupero password
	\item Gestione errori
\end{itemize}

\subsection{Thiella Eleonora}

\begin{itemize}
	\item CSS Desktop;
	\item CSS Mobile;
	\item CSS Stampa;
	\item HTML
	\item Accessibilità
\end{itemize}

\end{document}