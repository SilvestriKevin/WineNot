\input{StyleLatex/global.tex}
\input{StyleLatex/layout.tex}
\input{res/local}

\title{\textbf{Relazione progetto TecWeb}}
\author{WineNot}

\date{1 Febbraio 2018}

\begin{document}

%\maketitle

\makeFrontPage

\tableofcontents

\newpage

\section{Abstract}

Lo scopo del progetto sviluppato è quello di implementare un sito web per un
concessionario, nella quale un utente pu visualizzare tutte le auto disponibili, sia nuove che usate, gestire le proprie riparazioni e prenotare dei tesdrive. La principale mansione del concessionario 
è la vendita di auto, sia nuove che usate. Dreamcars inoltre offre assistenza post vendita a qualsiasi modello da lui
venduto. Non sarà quindi possibile eseguire un intervento di assistenza su un'auto
non venduta specicatamente dalla concessionaria. L'utente pu interagire sul sito
a seconda della sua affiliazione. Un utente non registrato, o che non ha effettuato
il login, potrà solo visualizzare un elenco di auto sia nuove che usate, anche sec-
ondo a dei parametri da lui scelti direttamente nella pagine degli annunci tramite
l'apposita funzione di ricerca. Un utente registrato, e che ha effettuato il login,
pu visualizzare lo stato degli interventi delle sue auto direttamente dalla sua area
personale. Da questa area personale l'utente pu, in aggiunta, modicare le infor-
mazioni base del suo account. Sempre da loggato, un utente pu prenotare un test
drive per unauto da lui scelta.
Un dipendente pu confermare i test drive prenotati dagli utenti, e visualizzare le
riparazioni delle auto modicandone eventualmente lo stato. Ogni dipendente in-
oltre pu inserire, rimuovere o modicare degli annunci, che compariranno poi nelle
pagine relative alle auto nuove o usate del sito.

\subsection{Utenti destinatari}

Il sito è destinato a qualsiasi utente cui scopo è quello di consultare, ed eventual-
mente prenotare un test drive di una specica auto. Una volta acquistata un'auto
direttamente in concessionaria, si accederà ad un ecosistema di assistenza secondo
il quale, sempre direttamente dal sito web, si potranno visualizzare gli interventi
dei propri autoveicoli.

\subsection{Gestione dei dati}

Il sito presenta dei contenuti modicabili da parte di utenti e di dipendenti (con
ruolo simile a quello di un amministratore). Qui di seguito sono riportati nello
specico le modiche possibili da parte di ciascuna categoria di utilizzatore:

\subsubsection{Utenti}

Previo login un utente pu vedere lo stato di avanzamento delle assistenze sulle
auto di sua proprietà, modicare le sue informazioni personali quali numero di
telefono, mail associata e password dell'account. Un cliente pu inoltre prenotare
un testdrive di una specica auto tramite l'apposito form sul sito. Sarà poi compito
di un dipendente confermare il test drive richiesto dall'utente.

\subsection{Amministratori}

Previo login un dipendente pu visualizzare, modicare o rimuovere le riparazioni
con i relativi interventi a cui è stata sottoposta qualsiasi auto in assistenza. Una
volta effettuato l'accesso all'area personale, un dipendente pu modificare le sue
informazioni di registrazione, quali telefono, mail e password. Sempre dalla pagina
di gestione account, un dipendente potrà confermare i testdrive richiesti dai clienti,
modicandone opportunamente date, ore e macchina selezionata. Un dipendente
pu inoltre aggiungere, modicare o rimuovere degli annunci.

\section{Fasi di progettazione}

Lo sviluppo del progetto si è suddiviso in varie fasi, raggruppabili nelle seguenti
categorie:

\subsection{Analisi dei requisiti}

Il gruppo si è pi volte riunito per analizzare ( e progettare ) dei requisiti che
potessero soddisfare ed essere il pi vicini possibili alla realtà di una concession-
aria. I requisiti presi in considerazione sono stati bacino di utenza, funzionalità,
e la visualizzazione ed interazione delle pagine web da parte di utenti e dipen-
denti. Il gruppo ha voluto dare una netta separazione tra dipendenti ed utenti, che si è ripercossa anche nei rispettivi permessi di modica del sito web. Inoltre,
abbiamo preso spunto da alcuni concessionari realmente esistenti, come quello di
Carraro Group ( http://www.gruppocarraro.it/ ) e Ceccato automobili ( http://
www.ceccatoautomobili.it/ ).

\subsection{Realizzazione ed implementazione delle funzionalità}

Ogni funzionalità implementata nel sito è frutto di un attento studio attraverso
esclusive sessioni di lavoro.
Il sito web si appoggia innanzitutto su una base di dati dove si vanno a memoriz-
zare tutte le informazioni viste a video, quali dati sulle auto, clienti, dipendenti,
riparazioni e testdrive. Le informazioni su questa base di dati vengono ricercate,
estrapolate e rielaborate attraverso speciali pagine, cui interno contengono delle
query che interrogano direttamente il database attraverso una connessione dedi-
cata.
Particolare attenzione è stata inoltre riservata alla visualizzazione del sito su
dispositivi mobile, dove risulta ottimizzata ogni pagina direttamente interagibile
dall'utente.

\subsection{Fase di test}

Ogni funzionalità implementata nel sito web è stata oggetto di numerosi test per
vericarne il suo effettivo funzionamento. Gli strumenti utilizzati sono stati W3C
Markup Validator, Total Validator, XHTML validator e browserstack.com.

\section{Struttura del progetto}

Causa enorme complessità del progetto, i le che compongono il sito sono stati
raggruppati e situati in 5 diverse cartelle:

\begin{itemize}
	\item \textbf{CARTELLA "css":} Questa cartella contiene tutti i file riguardanti il css, opportunamente diviso in tre file, per desktop, mobile e stampa;
	\item \textbf{CARTELLA "html":} Questa cartella contiene i file html del sito web;
	\item \textbf{CARTELLA "img":} Questa cartella contiene tutte le immagini del sito
	web;
	\item \textbf{CARTELLA "include":}
	\item \textbf{CARTELLA "js":} Questa pagina contiene tutti gli script javascript
	del progetto, quali "javascript.js" (script per slideshow e controlli sull'input) e "mappa.js" (script per visualizzare la mappa sulla pagina "info" del sito
	web);
	\item \textbf{CARTELLA "php":} Questa cartella contiene i file php del sito
	web;
	\item \textbf{CARTELLA "utility":} Questa cartella contiene un eseguibile utile per verificare il corretto funzionamento del recupero password;
\end{itemize}

\section{Gestione dei dati}

Il sito web si appoggia su una base di dati di tipo MySQL. Il database immagazzina
tutti i dati inseriti negli appositi form del sito web, rendendoli disponibili ad ogni
query eseguita in una pagina. Tutte le modiche avvengono tramite query MySQL
lanciate da pagine web PHP. La struttura del database è come segue:

\begin{itemize}
	\item \textbf{TABELLA "Annata":} in questa tabella vengono immagazzinate tutte le
	informazioni sulle annate dei vini presenti nell'enoteca. La tabella è strutturata come segue:
	\begin{itemize}
		\item \textbf{Anno:} indica l'anno ed è univoco;
		\item \textbf{Descrizione:} indica ;
		\item \textbf{Qualita:} ;
		\item \textbf{Migliore:} .
	\end{itemize}	
	\item \textbf{TABELLA "Utenti":} in questa tabella vengono immagazzinate tutte le
	informazioni sugli utenti presenti nell'enoteca. La tabella è strutturata
	come segue:
	\begin{itemize}
		\item \textbf{Id\_user:} indica ed è univoco;
		\item \textbf{Nome:} indica ;
		\item \textbf{Username:} ;
		\item \textbf{Password:} ;
		\item \textbf{Email:} ;
		\item \textbf{Admin:} .
	\end{itemize}
	\item \textbf{TABELLA "Vini":} in questa tabella vengono immagazzinate tutte le
	informazioni sui vini presenti nell'enoteca. La tabella è strutturata
	come segue:
	\begin{itemize}
		\item \textbf{Id\_wine:} indica ed è univoco;
		\item \textbf{Nome:} indica ;
		\item \textbf{Tipologia:} ;
		\item \textbf{Descrizione:} ;
		\item \textbf{Denominazione:} ;
		\item \textbf{Annata:} ;
		\item \textbf{Vitigno:} ;
		\item \textbf{Abbinamento:} ;
		\item \textbf{Degustazione:} ;
		\item \textbf{Gradazione:} ;
		\item \textbf{Formato:} .
	\end{itemize}
\end{itemize}

\section{Struttura sito}

Alcune funzionalità del sito sono usufruibili previa registrazione e login. Qui di
seguito sono elencate tutte le funzionalità del sito, raggruppate secondo le seguenti
nomenclature:
Everyone: identica qualsiasi utente ( anche dipendente ), registrato e non.
Registered: identica qualsiasi utente registrato e che ha eseguito il login al sito
web. Employee: identica qualsiasi dipendente registrato e che ha eseguito il login
al sito web.

\subsection{Funzionalità utenza usufruibile}

\begin{center}
	\begin{tabular}{r|l}
		Esplorazione del sito & Everyone \\
		Ricerca vini & Everyone \\
		... & ... \\
	\end{tabular}
\end{center}

\section{Adattabilità ed usabilità del sito}

La correttezza della lettura del sito web è facilitata dall'introduzione di elementi
tabindex, e la forte presenza di colori con contrasto elevato facilita la lettura del
sito web anche da persone con capacità visive limitate. Il sito web inoltre è realizzato tramite layout responsive e nel completo rispetto dei vigenti standard di
accessibilità.
Non risulta inoltre mai necessario alcuno scorrimento orizzontale in quanto il sito
web si adatta alla dimensione di ogni schermo da dove viene visualizzato. Nel
sito web sono presenti alcuni elementi per aiutare l'utente nella navigazione, quali
utilizzo di breadcrumb e colori diversi per ogni link visitato. Non sono inoltre
presenti contenuti a rischio di epilessia o che possano indurre l'utente in uno stato
confusionale.

\section{Compatibilità browser}

Il presente sito web è stato testato su vari browser in varie piattaforme tramite l'utilizzo del sito browserstack.com. Dai test effettuati, non sono emerse problematiche che possano in
uire negativamente sulla navigazione o sullusabilità del sito.
Di seguito si riporta un breve riepilogo delle principali piattaforme testate:
-Internet Explorer 8.0: Il sito è visualizzato ed interagibile correttamente -
Firefox 30: Il sito è visualizzabile ed interagibile correttamente -Safari iOS: Il sito
è visualizzabile ed interagibile correttamente -Google Chrome 60: Il sito è visual-
izzabile ed interagibile correttamente -Google Chrome 40: Il sito è visualizzabile
ed interagibile correttamente -Opera 12.12: Il sito è visualizzabile ed interagibile
correttamente

\section{Organizzazione del gruppo di lavoro}

La seguente tabella indica quali sono state le principali mansioni di ogni compo-
nente del gruppo. Ogni componente, tuttavia, ha contribuito almeno in parte ad
ogni fase di sviluppo del progetto.

\subsection{Pirlog Cristian}

\subsection{Silvestri Kevin}

\subsection{Thiella Eleonora}

\begin{itemize}
	\item CSS Desktop;
	\item CSS Mobile;
	\item HTML
	\item Javascript
	\item Validazione
	\item Creazione database
	\item Aiuto query php, parte controlli php ed aiuto pagine php utenti
	\item Stesura relazione
	\item PHP
	\item Ricerca vini
	\item Controlli PHP
	\item PHP: gestione sessioni e gestione connessioni
	\item Gestione login
	\item Gestione recupero password
	\item Gestione errori
\end{itemize}

\end{document}